\documentclass[letterpaper, 12pt, oneside]{article}%especificaciones del documento
\usepackage{amsmath}%paquete para escribir expresiones matemámaticas
\usepackage{graphicx}%paquete para poder incluír imagenes en el documento
\usepackage{xcolor} %paquete de LaTex para poder poner otro texto
\graphicspath{{Imagenes/}}%directorio de la imagen, este lo cambian por el directorio en el que ustedes guardaron su imagen 1.png
\usepackage[utf8]{inputenc} %para poder poner acentos
\usepackage{listings}

	\title{\Huge Taller de Herramientas Computacionales}
	%\title{\Huge \colorbox{magenta}{Taller de Herramientas computacionales}} %De esta forma con colorbox pone el texto dentro de una "caja" de color.
	\author{Josué Artemio Hernández Rodríguez}%autor del escrito
	\date{21/Enero/19}%fecha del escrito

\begin{document}%inicia el documento
\maketitle
%\vfill %Para rellenar el espacio y colocar hasta abajo de la pagina el siguiente texto, imagen.
\begin{center}%inicia centrado
\includegraphics[scale=.6]{3.jpg}%del lado izquierdo se muestra el tamaño de la imagen, del derecho se escribe el nombre de la imagen a incluir en el texto
\end{center}%termina el centrado para la imagen
\newpage%crea una nueva página

\title{\huge Preguntas: 24 de enero de 2019\\}%titulo2 \\ sirven para saltar una linea.


\begin{enumerate}%Inicio de númeración para enlistar las cosas vistas en clase.
	\item ¿Como haces para darle un formato de matriz a una lista? \\
	 Se podria utilizar \textit{print (numpy.matrix(L))}, importando previamente ese modulo
	 y le dara un formato como de una matriz.
	
	\item ¿Como instalas numpy?\\
	Para instalar numpy basta con escribir esto en la terminal: \textit{sudo pip install numpy}\\ esto en el caso de fedora, pero en otras distribuciones de linux funciona de forma similar
	\item ¿Que es una variable local?\\
	Es aquella variable que existe mientras hago el llamado a la función o mientras se este ejecutando. Y queda siempre dentro de la función. 
	
	\item ¿Que es elif y como funciona?
	Es una condicional, es igual a tener un \textit{if} y dentro un \textit{else}. Lo que hace es verificar lo que este el bucle y si no pasa a otro \textit{elif} o \textit{else}. Pueden estar varios \textit{elif} debajo uno de otro verificando condiciones.
	 
	
	
	
	
	
\end{enumerate}%finaliza el enlistado principal


\end{document}%termina el documento