\documentclass{book}
\usepackage[spanish]{babel}
\usepackage[utf8]{inputenc}
\usepackage{biblatex}
\usepackage{hyperref}


\title{Taller de Herramientas Computacionales}
\author{Josue Artemio Hernandez Rodriguez}
\date{17/Enero/2019}


\begin{document}
\maketitle
%Aqui inicia el indice de contenido del texto
\tableofcontents
\section*{Introducción} Este libro es para formalecer el conocimiento de la materia de herramientas computacionales.
\url{www.google.com.mx}\\
\hyperref[Google]{www.google.com}

\chapter{Uso Basico de Linux}
\section{Distribuciones de Linux}
\section{Comandos}
\chapter{Introducción a Latex}
\chapter{Introducción a Python}

\begin{verbatim}
#!/usr/bin/python2.7 
# -*- coding: utf-8 -*-
'''
Josué Artemio Hernéndez Rodríguez, 316060064
Taller de Herramientas Computacionales

Le dimos atributos de ejecución a un programa de python y ademas dentro del código
agregamos que interprete va utilizar
'''

x = 10.5;y = 1.0/3;z = 15.3
#x,y,z = 10.5,1.0/3,15.3

H = """
El punto en R3 es:
(x,y,z)=(%.2f,%g,%G)
""" % (x,y,z)

print H

G="""
El punto en R3 es:
(x,y,z)=({laX:.2f},{laY:g},{laZ:G})
""".format(laX=x,laY=y,laZ=z)

print G


import math as m
from math import sqrt
from math import sqrt as s
from math import *  # no se recomienda
x=16
x = input("Cual es el valor al que le quieres\n" +
" calcular la raiz: ")
print "la raiz cuadrada de %.2f es %f" % (x,m.sqrt(x))
print sqrt(16.5)
print s(16.5)
\end{verbatim}






%Aqui inician los capitulos del libro
\chapter{Introducción a Latex}
\chapter{Introducción a Python}
\section*{Orientacion a objetos}

\begin{thebibliography}{9}
	\bibitem{Computacion}
	Autor blah blah blah
	\textit{cualquier cosa}
	blah blah blah 2019
\end{thebibliography}
	
\end{document}