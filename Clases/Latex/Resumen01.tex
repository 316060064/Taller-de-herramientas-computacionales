\documentclass[letterpaper, 12pt, oneside]{article}%especificaciones del documento
\usepackage{amsmath}%paquete para escribir expresiones matemámaticas
\usepackage{graphicx}%paquete para poder incluír imagenes en el documento
\usepackage{xcolor} %paquete de LaTex para poder poner otro texto
\graphicspath{{Imagenes/}}%directorio de la imagen, este lo cambian por el directorio en el que ustedes guardaron su imagen 1.png
\usepackage[utf8]{inputenc} %para poder poner acentos0

	\title{\Huge Taller de Herramientas Computacionales}
	%\title{\Huge \colorbox{magenta}{Taller de Herramientas computacionales}} %De esta forma con colorbox pone el texto dentro de una "caja" de color.
	\author{Josué Artemio Hernández Rodríguez}%autor del escrito
	\date{07/01/19}%fecha del escrito

\begin{document}%inicia el documento
\maketitle
%\vfill %Para rellenar el espacio y colocar hasta abajo de la pagina el siguiente texto, imagen.
\begin{center}%inicia centrado
\includegraphics[scale=0.2]{2.png}%del lado izquierdo se muestra el tamaño de la imagen, del derecho se escribe el nombre de la imagen a incluir en el texto
\end{center}%termina el centrado para la imagen
\newpage%crea una nueva página

\title{\Huge Introducción al curso\\}%titulo2 \\ sirven para saltar una linea.

Lo que vimos en clase fue:%letras de color azul
\begin{enumerate}%Inicio de númeración para enlistar las cosas vistas en clase.
	\item Distribuciones de Linux: %item sirve enlistar el elemento, este es el primer elemento enumerado.
	\begin{enumerate}%Inicia otro enlistado dentro del listado anterior.
		\item Fedora
		\item Linux
		\item Ubuntu
	\end{enumerate}%finaliza el enlistado.
	\item Comandos de Bash%Segundo elemento enumerado.
	\begin{itemize}%comienza el enlistado pero itemize a diferencia de enumerate enlista sin un orden secuencial (es decir no utiliza números, ni letras)
		\item set
		\item cd: nos dirige a un directorio
		\item mkdir: Crea un directorio
		\item mkdir -p Crea un arbol de directorios
		\item ls: Muestra información de un archivo o directorio especifico, pero si agregamos -l muestra permisos
		\item File: Nos dice tipo de archivo es
		\item uname-a: Nos dice que arquitectura es nuestro sistema, muy util para saber si estamos aprovechando las ventajas de un CPU X64
		\item cmod "   ": Modifica los permisos de un archivo o directorio en especifico
				
	\end{itemize}%finaliza enlistado con itemize
	\item Como vamos a trabaar en el curso
		\begin{itemize}
			\item Libro de python disponible en la bibliografia de la materia, y gratuito en www.bibliotecas.unam.mx
			\item Resumenes de las clases en LaTeX
			\item Examenes cada viernes
			\item Ver en que aplicar lo aprendido
		\end{itemize}
	\item \textbf{Tarea}
	\\
	\textbf{Verificar todos los comandos}
	
	
	
\end{enumerate}%finaliza el enlistado principal
	


\end{document}%termina el documento