\documentclass[letterpaper, 12pt, oneside]{article}%especificaciones del documento
\usepackage{amsmath}%paquete para escribir expresiones matemámaticas
\usepackage{graphicx}%paquete para poder incluír imagenes en el documento
\usepackage{xcolor} %paquete de LaTex para poder poner otro texto
\graphicspath{{Imagenes/}}%directorio de la imagen, este lo cambian por el directorio en el que ustedes guardaron su imagen 1.png
\usepackage[utf8]{inputenc} %para poder poner acentos0

	\title{\Huge Taller de Herramientas Computacionales}
	%\title{\Huge \colorbox{magenta}{Taller de Herramientas computacionales}} %De esta forma con colorbox pone el texto dentro de una "caja" de color.
	\author{Josué Artemio Hernández Rodríguez}%autor del escrito
	\date{08/01/19}%fecha del escrito

\begin{document}%inicia el documento
\maketitle
%\vfill %Para rellenar el espacio y colocar hasta abajo de la pagina el siguiente texto, imagen.
\begin{center}%inicia centrado
\includegraphics[scale=0.2]{2.png}%del lado izquierdo se muestra el tamaño de la imagen, del derecho se escribe el nombre de la imagen a incluir en el texto
\end{center}%termina el centrado para la imagen
\newpage%crea una nueva página

\title{\Huge Clase en laboratorio\\}%titulo2 \\ sirven para saltar una linea.

Lo que vimos en clase fue:%letras de color azul
\begin{enumerate}%Inicio de númeración para enlistar las cosas vistas en clase.
	\item Algunos conceptos y software utilizado en clase%item sirve enlistar el elemento, este es el primer elemento enumerado.
		\begin{itemize}
			\item bash: Define la forma en como esta funcionando el sistema
			\item Path : Ruta donde va buscar los binarios
			\item Tigervnc: Sirve para mirar remotamente la pantalla o área de trabajo de otra PC en la misma red
			\item github: Pagina de alojamiento y servicio de repositorios de git
			\item git: Cliente de github para fedora (linux) el cual usaremos para mantener actualizado al dia los repositorios en la nube
		
		\end{itemize}
	
	\item Comandos de Bash%Segundo elemento enumerado.
	\begin{itemize}%comienza el enlistado pero itemize a diferencia de enumerate enlista sin un orden secuencial (es decir no utiliza números, ni letras)
		\item file /tmp/* Muestra que tipo de archivos hay en esa ruta
		\item set | less : Muestra paginando el contenido, para pasar a la pagina siguiente se oprime "RePág" o retrocede con "AvPág"
		
				
	\end{itemize}%finaliza enlistado con itemize
	\item Còmo instalar y configurar tigervnc
		\begin{enumerate}
			\item En la terminal, escribir "dnf install tygervnc" y esperar que se instale tigervnv
			\item Escribir la ip: "192.168.13.125"
		\end{enumerate}
	\item Còmo instalar y configurar git
		\begin{enumerate}
			\item Teclear en la terminal "sudo yum install git", esperar a que se instalen los paquetes necesarios...
			\item Teclear  git config --global user.email "tuemail@dominio.com"
			\item Teclear  git config --global user.name "tuusuario"      
			\item Teclear  git init
			\item Teclear  git clone "url
			\item Teclear git add *
			\item Teclear git commit (Se escribe un comentario sobre lo que hicimos, para salir se sigue esta secuencias: ESC, :wq)
			\item Escribir el usuario de github
			\item Escribir la contraseña
		\end{enumerate}
	
	
	
	
\end{enumerate}%finaliza el enlistado principal
	


\end{document}%termina el documento