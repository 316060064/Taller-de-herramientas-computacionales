\documentclass[letterpaper, 12pt, oneside]{article}%especificaciones del documento
\usepackage{amsmath}%paquete para escribir expresiones matemámaticas
\usepackage{graphicx}%paquete para poder incluír imagenes en el documento
\usepackage{xcolor} %paquete de LaTex para poder poner otro texto
\graphicspath{{Imagenes/}}%directorio de la imagen, este lo cambian por el directorio en el que ustedes guardaron su imagen 1.png
\usepackage[utf8]{inputenc} %para poder poner acentos
\usepackage{listings}

	\title{\Huge Taller de Herramientas Computacionales}
	%\title{\Huge \colorbox{magenta}{Taller de Herramientas computacionales}} %De esta forma con colorbox pone el texto dentro de una "caja" de color.
	\author{Josué Artemio Hernández Rodríguez}%autor del escrito
	\date{22/Enero/19}%fecha del escrito

\begin{document}%inicia el documento
\maketitle
%\vfill %Para rellenar el espacio y colocar hasta abajo de la pagina el siguiente texto, imagen.
\begin{center}%inicia centrado
\includegraphics[scale=.6]{3.jpg}%del lado izquierdo se muestra el tamaño de la imagen, del derecho se escribe el nombre de la imagen a incluir en el texto
\end{center}%termina el centrado para la imagen
\newpage%crea una nueva página

\title{\huge Preguntas: 22 de enero de 2019\\}%titulo2 \\ sirven para saltar una linea.


\begin{enumerate}%Inicio de númeración para enlistar las cosas vistas en clase.
	\item ¿Por qué en python se les denomina a las listas como "listas circulares"? \\
	 Porque si especificas un índice negativo toma valores de derecha a izquierda.
	
	\item ¿Qué son las variables?\\
	Las variables son identificadores de un valor especifico, y sirve para representar alguna expresion muy larga y asi no tener que volver a escribirla en un modulo o bloque que la necesite, o simplemente para resguardar un valor.
	
	\item ¿Cuál es la diferencia de \textit{for i in range():} y \textit{for i in lista:}?\\
	Que la primera toma en cuenta los índices y la segunda forma toma la posición.
	
	\item ¿Qué resulta de estos códigos de python?\\
	\begin{itemize}
		\item \textit{A[2:]} : Regresa valores a partir del índice 2
		\item \textit{A[1:3]}; Pone como rango de elementos a mostrar al índice 1 y 3, pero no se muestra el 3, digamos que es un intervalo cerrado/abierto.
		\item \textit{A[:3]} : Muestra todos los valores que pertenecen a cada indice en A, pero como limite al 3, sin mostrarlo
		\item \textit{[1:-1]}  : Regresa todo, excepto el primero y el ultimo.
	\end{itemize}
	
	\item ¿Qué resulta tras la compilación de este código: {\textit{n=12; gradosC = [-5 + i*0.5 for i in range(n)]}}?
	
	Crea una lista de forma corta o por comprension, y el resultado es una sucesión de numeros enteros, con un rango de -5 a 0.5, y va aumentando en 1/2.
		
	
	\item ¿Que uso se le da a: \begin{verbatim}
	\begin{frame}[fragile] ?
	\end{verbatim}
	
	Se utiliza para especificar que en esa zona se colocara un begin{verbatin} para escribir código de programación de python u otros.
	
	
	\item ¿Para que sirve el tipo de documento beamer en \LaTeX?
	Este tipo de documento se utiliza para crear presentaciones.
	
	 
	
	
	
	
	
\end{enumerate}%finaliza el enlistado principal


\end{document}%termina el documento