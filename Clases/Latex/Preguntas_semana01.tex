\documentclass[letterpaper, 12pt, oneside]{article}%especificaciones del documento
\usepackage{amsmath}%paquete para escribir expresiones matemámaticas
\usepackage{graphicx}%paquete para poder incluír imagenes en el documento
\usepackage{xcolor} %paquete de LaTex para poder poner otro texto
\graphicspath{{Imagenes/}}%directorio de la imagen, este lo cambian por el directorio en el que ustedes guardaron su imagen 1.png
\usepackage[utf8]{inputenc} %para poder poner acentos0

	\title{\Huge Taller de Herramientas Computacionales}
	%\title{\Huge \colorbox{magenta}{Taller de Herramientas computacionales}} %De esta forma con colorbox pone el texto dentro de una "caja" de color.
	\author{Josué Artemio Hernández Rodríguez}%autor del escrito
	\date{13/01/19}%fecha del escrito

\begin{document}%inicia el documento
\maketitle
%\vfill %Para rellenar el espacio y colocar hasta abajo de la pagina el siguiente texto, imagen.
\begin{center}%inicia centrado
\includegraphics[scale=0.2]{2.png}%del lado izquierdo se muestra el tamaño de la imagen, del derecho se escribe el nombre de la imagen a incluir en el texto
\end{center}%termina el centrado para la imagen
\newpage%crea una nueva página

\title{\huge Preguntas: Semana 1\\}%titulo2 \\ sirven para saltar una linea.

La semana se resume en las siguientes preguntas:%letras de color azul
\begin{enumerate}%Inicio de númeración para enlistar las cosas vistas en clase.
	\item Menciona las distribuciones de linux que conozcas \\
	 Fedora, ubuntu, mint	
	
	\item ¿Cómo se llama el elemento que parpadea en la pantalla de la terminal?\\
	Pront
	\item ¿Cómo haces para ir a Home directamente en la terminal?\\
	Usando el comando cd
	\item ¿A qué se refiere esta linea de comando "file /bin/bash?\\
	Muestra la informacion del archivo bash
	\item ¿Como haces para saber que arquitectura es la distribución de inuz que estas usando?\\
	Con el comando uname -a
	\item ¿Si tienes una duda, como abres el manual de un comando en especifico?\\
	Con man seguido del comando que quiero saber utilizar
	\item ¿Qué es un ssh?\\
	Es un interprete de comandos seguros
	\item ¿Qué es Bash?\\
	Define la forma en como esá funcionando el sistema
	
	\item ¿Cómo instalas tigervnc y para que sirve?\\
	Con el comando "dnf yum install tigervnc" y sirve para ver remotamente en mi pantalla lo que se trasmite desde otra desde la misma red.
	\item ¿Para qué sirve git pull?\\
	Para obtener una actualizacion de mis datos alojados en mi ciente de github y asi evitar inconsistencias
	
	\item Menciona dos o mas editores de texto en Linux que conozcas\\
	vi, nano, y gedit
	\item ¿Cómo haces para crear un árbol de directorio en una sola linea de comando?\\
	Con mkdir -p seguido de la rama de directorios a crear
	\item ¿Como guardas el historial de la terminal?\\
	Con \$ history > Carpeta/Subcarpeta/Comandos.txt
	\item ¿Qué es idle?\\
	Es un IDE, que quiere decir entorno de desarrollo integrado, y sirve para crear un script de instrucciones para python
	
	\item ¿Cómo realizas comentarios en python?\\
	Con \# y con '''  '''
	\item Si tienes una variable con una cadena, ¿cómo haces para mostrar su contenido con print?\\
	Debo poner print '\%s' \%(c1) donde c1 es otra variable especificada
	\item ¿Qué son los modulos de python?\\
	Son bibliotecas de funciones
	\item ¿En python a que se refiere \%g o  \%f, \%e o \%E, y \%d?\\
	En primera instancia, \% denota que se va sustituir por el valor de una variable ya definida por el ususario, y \%g o  \%f se refieren a que se va expresar en flotante en su formato mas corto o simplemente como un flotante; \%e o \%E se refiere a que se va sustituir por una variable y se expresara en formato científico, y la diferencia radica en que se expresara con minisucula o mayusculo a consideración del usuario; y \%d se refiere a que se expresara como un entero.
	\item En TeXstudio, menciona como especificas el tipo de documento que vas a realizar y parametros de formato\\
	Con documentclass[parametros]\{tipo\}
	\item Menciona el paquete que tienes que usar para poder incluir imagenes en el documento en TeXstudio\\
	Se utiliza el paquete usepackage\{graphicx\} y ya que vas a colocar la imagen se escribe includegraphics[parametros]\{nombrearchivo.png\}
	\item ¿Cómo instalas TeXstudio?\\
	Desde la terminal escribo en el pront \$ dnf install texstudio
	
	
	
	
\end{enumerate}%finaliza el enlistado principal


\end{document}%termina el documento