\documentclass[letterpaper, 12pt, oneside]{article}%especificaciones del documento
\usepackage{amsmath}%paquete para escribir expresiones matemámaticas
\usepackage{graphicx}%paquete para poder incluír imagenes en el documento
\usepackage{xcolor} %paquete de LaTex para poder poner otro texto
\graphicspath{{Imagenes/}}%directorio de la imagen, este lo cambian por el directorio en el que ustedes guardaron su imagen 1.png
\usepackage[utf8]{inputenc} %para poder poner acentos
\usepackage{listings}

	\title{\Huge Taller de Herramientas Computacionales}
	%\title{\Huge \colorbox{magenta}{Taller de Herramientas computacionales}} %De esta forma con colorbox pone el texto dentro de una "caja" de color.
	\author{Josué Artemio Hernández Rodríguez}%autor del escrito
	\date{23/Enero/19}%fecha del escrito

\begin{document}%inicia el documento
\maketitle
%\vfill %Para rellenar el espacio y colocar hasta abajo de la pagina el siguiente texto, imagen.
\begin{center}%inicia centrado
\includegraphics[scale=.6]{3.jpg}%del lado izquierdo se muestra el tamaño de la imagen, del derecho se escribe el nombre de la imagen a incluir en el texto
\end{center}%termina el centrado para la imagen
\newpage%crea una nueva página

\title{\huge Preguntas: Semana 1\\}%titulo2 \\ sirven para saltar una linea.

La semana 2 se resume en las siguientes preguntas:%letras de color azul
\begin{enumerate}%Inicio de númeración para enlistar las cosas vistas en clase.
	\item ¿Qué es un algoritmo? \\
	 Es un conjunto finito de instrucciones para resolver
	 un problema especifico.
	
	\item Cuando decimos que "range() es una lista inmutable de números enteros", ¿a qué se refiere con inmutable?\\
	A que a diferencia de las listas, los range no se pueden modificar
	\item ¿Qué es una tupla?\\
	Es una lista que ya no puedes modificar, a diferencia de las listas que si es posible, pero si es posible extraer sus valores
	
	\item ¿Para que se utiliza {\textit{enumerate}}?\\
	Te da el valor del índice y ademas el calor del indice dentro de la lista
	
	\item ¿Qué resulta tras la compilación de este código: {\textit{n=12; gradosC = [-5 + i*0.5 for i in range(n)]}}?
	
	Crea una lista de forma corta o por comprension, y el resultado es una sucesión de numeros enteros, con un rango de -5 a 0.5, y va aumentando en 1/2.
		
	
	\item ¿Como funciona y para que sirve: {\textit{zip() :}}?\\
	
	Toma como argumento dos o más objetos iterables y regresa un nuevo iterable cuyos elementos son tuplas que contienen un elemento de cada uno de los objetos(listas por ejemplo).
	
	
	\item Menciona que hace el siguiente código: \begin{verbatim}
for i, c in enumerate(gradosC):
    gradosC[i] = c +5
    print i, c
	\end{verbatim}
	Al elemento de cada indice de gradosC le suma 5, e imprime el ìndice y c, que son los valores aumentados en 5.
	
	 
	
	
	
	
	
\end{enumerate}%finaliza el enlistado principal


\end{document}%termina el documento