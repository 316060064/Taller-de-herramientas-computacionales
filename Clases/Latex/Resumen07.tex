\documentclass[letterpaper, 12pt, oneside]{article}%´
\usepackage{amsmath}%paquete para escribir expresiones matemámaticas
\usepackage{graphicx}%paquete para poder incluír imagenes en el documento
\usepackage{xcolor} %paquete de LaTex para poder poner otro texto
\graphicspath{{Imagenes/}}%directorio de la imagen, este lo cambian por el directorio en el que ustedes guardaron su imagen 1.png
\usepackage[utf8]{inputenc} %para poder poner acentos
\usepackage{listings}

\title{\Huge Taller de Herramientas Computacionales}
%\title{\Huge \colorbox{magenta}{Taller de Herramientas computacionales}} %De esta forma con colorbox pone el texto dentro de una "caja" de color.
\author{Josué Artemio Hernández Rodríguez}%autor del escrito
\date{15/01/19}%fecha del escrito

\begin{document}%inicia el documento
	\maketitle
	%\vfill %Para rellenar el espacio y colocar hasta abajo de la pagina el siguiente texto, imagen.
	\begin{center}%inicia centrado
		\includegraphics[scale=0.2]{2.png}%del lado izquierdo se muestra el tamaño de la imagen, del derecho se escribe el nombre de la imagen a incluir en el texto
	\end{center}%termina el centrado para la imagen
	\newpage%crea una nueva página
	
	\title{\Huge Funciones para resolver problemas en python\\}%titulo2 \\ sirven para saltar una linea.
	
	Lo que vimos en clase fue:%letras de color azul
	\begin{enumerate}%Inicio de númeración para enlistar las cosas vistas en clase.
		\item Algunos conceptos y software utilizado en clase%item sirve enlistar el elemento, este es el primer elemento enumerado.
		\begin{itemize}
			\item Partes globales son: definir valores iniciales, parte que procesa la información, y parte del resultado.	
			\item La diferencia dentre if y while es que if se realiza una vez y while 0 o más veces		
			\item Python sirve para realizar calculo simbolico y numerico
			\item Tex se encarga de organizar un documento, y LaTeX es una herramienta de creación de documentos
		\end{itemize}
		
		\item Comandos de python%Segundo elemento enumerado.
		\begin{itemize}%comienza el enlistado pero itemize a diferencia de enumerate enlista sin un orden secuencial (es decir no utiliza números, ni letras)
			\item while : Es un bucle qie mientras la condicion no se cumpla sigue ejecutandose hasta que sea verdadero
			\item if : Condicional que si es verdadero ejecuta
			\item a = 5 : Es una asignación
			\item a == 5 : Es de comparación
			
			
			
		\end{itemize}%finaliza enlistado con itemize
		\item Comandos de LaTeX
		\begin{itemize}
			\item  
			\begin{lstlisting}  
\section{} 
			\end{lstlisting} Abre una sección nueva con un número
			\item 
			\begin{lstlisting}  
\section*{}
			\end{lstlisting} Abre una sección nueva sin número
			\item 
			\begin{lstlisting}  
$\alpha + \beta$\\ %\[  \]
\[\alpha + \beta\] 
			\end{lstlisting} Nos ayudan a representar las letras que representan a alpha y beta\\
			
			\begin{lstlisting}  
$x_{2}$
$x^{2}$
			\end{lstlisting}  Representa un subindice e indice respectivamente
			\item 
			\begin{lstlisting}  
$\frac{2}{3}$\\
$\frac{\frac{3}{4}}{\frac{2}{3}}$
			\end{lstlisting} Nos ayudan a representar fracciones e dos formatos distintos.
			\item
			\begin{lstlisting}  
$\sqrt{2} 
 \sqrt{3^2}^2$
			\end{lstlisting} Expresan raiz cuadrada de un numero y la otra la raiz cuadrada elevada a una potencia, en este caso 2
			\item
			\begin{lstlisting} 
$\int_{a}^{b} x^2$
			\end{lstlisting} Expresa la integral de "a" a "b", cuando x esta elevada al cuadrado
			\item
			\begin{lstlisting}  
$\partial x^2$
			\end{lstlisting} La parcial de x al cuadrado
			\item
			\begin{lstlisting} 
$3\quad 2$ 			
			\end{lstlisting} Coloca un espacio entre el 3 y el 2
			\item
			\begin{lstlisting} 
\usepackage{listings}			
			\end{lstlisting} Este paquete nos permite colocar código de programacion en nuestro texto
			
			
			
			
			
		\end{itemize}
		
		
		
		
	\end{enumerate}%finaliza el enlistado principal
	
	
	
\end{document}%termina el documento