\documentclass{beamer}
\usepackage[utf8]{inputenc}
\usepackage[spanish]{babel}
\usepackage{graphicx}
\graphicspath{{Imagenes/}}
%\usetheme{Antibes}
%\usetheme{AnnArbor}
%\usetheme{Berkeley}
%\usetheme{CambridgeUS}
%\usetheme{Goettingen}
%\usetheme{Bergen}
%\usetheme{Dresden}
%\usetheme{Hannover}
%\usetheme{Ilmenau}
%\usetheme{Berlin}
%\usetheme{Boadilla}
\usetheme{Darmstadt}




%Esto unicamente se utiliza para el tema Bergen
%\def\insertauthorindicator{¿Quien?}
%\def\insertdateindicator{Fecha}
%\title{Taller de Herramientas computacionales}
%\author{Josué Artemio Hernandez Rodríguez}
%\date{\today}

\title{Taller de Herramientas Computacionales}
\author{Josué Artemio Hernández Rodríguez}
\date{22/Enero/2019}


\begin{document}
	\maketitle
	
\begin{frame}
%\transboxin
\transblindshorizontal
	\frametitle{Mi Primera presentación en LaTeX}
	\begin{align*}
		\includegraphics[scale=0.15]{2.png}
	\end{align*}
	
\end{frame}	


\begin{frame}
\frametitle{Segunda diapositiva}
	Esta es mi segunda diapositiva
\end{frame}

\begin{frame}[fragile]
	\begin{verbatim}
#!/usr/bin/python2.7 
# -*- coding: utf-8 -*-
'''
Josué Artemio Hernéndez Rodríguez, 316060064
Taller de Herramientas Computacionales

Le dimos atributos de ejecución a un programa de python y ademas dentro del código
agregamos que interprete va utilizar
'''

x = 10.5;y = 1.0/3;z = 15.3
#x,y,z = 10.5,1.0/3,15.3

H = """
El punto en R3 es:
(x,y,z)=(%.2f,%g,%G)
""" % (x,y,z)

print H

G="""
El punto en R3 es:
(x,y,z)=({laX:.2f},{laY:g},{laZ:G})
""".format(laX=x,laY=y,laZ=z)

print G


import math as m
from math import sqrt
from math import sqrt as s
from math import *  # no se recomienda
x=16
x = input("Cual es el valor al que le quieres\n" +
" calcular la raiz: ")
print "la raiz cuadrada de %.2f es %f" % (x,m.sqrt(x))
print sqrt(16.5)
print s(16.5)

	\end{verbatim}
\end{frame}
	
	
\end{document}