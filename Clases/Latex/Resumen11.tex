\documentclass[letterpaper, 12pt,oneside]{article}
\usepackage{amsmath}
\usepackage{graphicx}
\usepackage{xcolor}
\graphicspath{{Imagenes/}}
\usepackage[utf8]{inputenc}
\usepackage{listings}

\title{\Huge Taller de Herramientas Computacionales}
\author{Josué Artemio Hernández Rodríguez}
\date{21/Enero/2019}

\begin{document}
	\maketitle
	\begin{center}
		\includegraphics[scale=0.2]{2.png}
	\end{center}

	\newpage
	
	\title{\huge \textit{El uso de listas en Python II}}\\
	
	El resumen del dia lunes 21 de Enero de 2019 abarca lo siguiente:
	\begin{enumerate}
		\item Conceptos
			\begin{itemize}
				\item El tipo range() es una lista inmutable de números enteros en sucesión aritmética.Inmutable significa que, a diferencia de las listas, los range no se pueden modificar, una sucesión artimética es una sucesión en la que la diferencia entre dos términos consecutivos es siempre la misma.
				
				\item Una tupla es una lista que ya no puedes modificar, a diferencia de las listas que si es posible, pero si es posible extraer sus valores de forma similar, es decir con su índice.
				\item Para evitar que se suban archivos innecesarios a nuestro repositorio de github es fundamental crear el archivo .gitignore, y en el editor de texto escribir las extensiones de archivo que no queremos que se suban cada que hagamos push al cliente de git, por ejemplo *.log, *.out, *.pdf, *.pyc, etc.
			\end{itemize}
		
	
		
		\item Código de python
			\begin{itemize}
				\item {\textit{enumerate}} : Te da el valor del índice y ademas el calor del indice dentro de la lista
				\item {\textit{[ ]}} : Los corchetes nos ayudan a acceder a un elemento dentro de una lista o para definir una lista.
				\item {\textit{a = lista(n,m,p)}} : Lo que hace es crear una lista que inicia en n y termina en m - 1, que va de p en p.
				\item {\textit{n=12; gradosC = [-5 + i*0.5 for i in range(n)]}} :  Es una forma corta de hacer una lista, \textit{por comprensión} en donde se define una n, despues la lista entre corchetes, donde por cada indice en n, lo multiplica por 0.5. Nos entrega como resultado:
				\begin{align*}
				[-5.0, -4.5, -4.0, -3.5, -3.0, -2.5, -2.0, -1.5, -1.0, -0.5, 0.0, 0.5]
				\end{align*}
				\item {\textit{zip() :}} Toma como argumento dos o más objetos iterables (idealmente cada uno de ellos con la misma cantidad de elementos) y retorna un nuevo iterable cuyos elementos son tuplas que contienen un elemento de cada uno de los iteradores originales.
					
				
				\item 
				\begin{lstlisting}
for i, c in enumerate(gradosC):
	gradosC[i] = c +5
	print i, c
				\end{lstlisting} Le suma 5 a cada elemento dentro de gradosC y te regresa los valores nuevos y el índice
				\item 
				\begin{lstlisting}
for i in range(len(L1)):
	L[i] += 5
				\end{lstlisting} Una forma diferente de hacer lo mismo que el anterior, pero usando range, ademas se toma como argumento la longitud de la lista. y a cada valor del indice le suma 5.
				
			\end{itemize}
		
		
	
		
	\end{enumerate}
	
	
	
	
	
	
	
	
	
	
	
\end{document}
