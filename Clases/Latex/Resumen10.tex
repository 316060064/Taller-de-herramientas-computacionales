\documentclass[letterpaper, 12pt,oneside]{article}
\usepackage{amsmath}
\usepackage{graphicx}
\usepackage{xcolor}
\graphicspath{{Imagenes/}}
\usepackage[utf8]{inputenc}
\usepackage{listings}

\title{\Huge Taller de Herramientas Computacionales}
\author{Josue Artemio Hernandez Rodriguez}
\date{18/Enero/2019}

\begin{document}
	\maketitle
	\begin{center}
		\includegraphics[scale=0.2]{2.png}
	\end{center}

	\newpage
	
	\title{\huge \textit{El uso de listas en Python}}\\
	
	El resumen del dia viernes 18 de Enero de 2019 abarca lo siguiente:
	\begin{enumerate}
		\item Conceptos
			\begin{itemize}
				\item Las listas en python son una herramienta muy interesante y potente que nos sirve para recolentar datos de manera robusta, estos datos adquieren la cualidad de objetos dentro de la lista.
				\item La diferencia de posición e índice es que la posición es un número n del 1 a un numero fínito, y el índice es (n-1).
			\end{itemize}
		
		\item Comandos de bash
			\begin{itemize}
				\item {\textit{idle ; emacs} }: El punto y coma separa comandos en bash, en este caso los comandos para abrir dos programas. No los abre juntos, si no que el segundo solo se abre al cerrarse el primero.
				\item {\textit{idle \& emacs}} : La separacion de dos comandos con \& representa que ambos se ejecutan al mismo tiempo. A diferencia del anterior, uno no depende del otro.
			\end{itemize}
		
		\item Código de python
			\begin{itemize}
				\item {\textit{type(cadena)}} : Type nos sirve para evaluar de que tipo es una variable, es decir, cadena, entero, flotante, etc.
				\item {\textit{bool(cadena)}} : Evalua si una variable es booleano, si lo es regresa {\textit{True}}, si no lo es {\textit{False}}. Si es cero regresa False y si es diferente de cero True.
				\item L = [] o L = [1.2.3.8.9] : Los corchetes repreentan una lista, la primera es una lista vacia, y la otra es una lista con los elemntos 1,2,3,8 y 9.
				\item L.append : Agrega elementos despues del último elemento dentro de la lista con variabel "L".
				\item len(L) : Nos dice de cuantos elementos esta conformada una lista.
				\item L.insert(3,"otra cadena"): Nos dice que  en el índice 3 agrega el elemento "otra cadena".
				\item L[3] : Nos muestra cual es el elemento con índice 3 dentro de la lista L
				\item len(L[5]) : Nos dice cual es la longitud del elemento con índice 5  dentro de la cadena L. Seria trivial si fuera una lista de solo unos cuantos numeros, asi que es util si la lista contiene como elementos otras listas.
				\item L.pop() : Saca un elemento de la lista L. Si no le especifico un índice dentro de () sacara el ultimo de la lista.
				\item L.extend : Agrega elementos independientes a una lista
				\item L[3] + L[len(L)-1] : Junta los elementos con indice 3 y el ultimo de la lista en uno solo, es decir si L[3]= ¿Cómo y L[len(L)-1] = estás?, entro de la lista L[1,2,3,"¿Cómo,45,62,"estás?"], nos da como resultado, "¿Cómoestás?"
				\item 
				\begin{lstlisting}
archivoLatex = ['libro.tex', 'libro.log', 'libro.pdf']
archivotex, archivolog, archivopdf = archivoLatex
				\end{lstlisting} ArchivoLatex es una lista con los elementos 'libro.tex', 'libro.log', 'libro.pdf'. Pero "archivotex, archivolog, archivopdf = archivoLatex" lo que hace es asignarle una variable a cada uno de esos elementos dentro de la lista. Es decir, al escribir archivotex no regresara como resultado 'libro.tex'
				\item 
				\begin{lstlisting}
for C in gradosC:
	print 'Elemento de la lista: ', C
				\end{lstlisting} Por cada indice en la lista gradosC, imprime "Elemento de la lista: + el elemento perteneciente a cada indice" 
				\item 
				\begin{lstlisting}
print 'La lista C tiene: ',len(C), 'elementos'
				\end{lstlisting} ~
				
			\end{itemize}
	
		
	\end{enumerate}
	
	
	
	
	
	
	
	
	
	
	
\end{document}
