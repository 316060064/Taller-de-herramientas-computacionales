\documentclass[letterpaper, 12pt, oneside]{article}%especificaciones del documento
\usepackage{amsmath}%paquete para escribir expresiones matemámaticas
\usepackage{graphicx}%paquete para poder incluír imagenes en el documento
\usepackage{xcolor} %paquete de LaTex para poder poner otro texto
\graphicspath{{Imagenes/}}%directorio de la imagen, este lo cambian por el directorio en el que ustedes guardaron su imagen 1.png
\usepackage[utf8]{inputenc} %para poder poner acentos
\usepackage{listings}

	\title{\Huge Taller de Herramientas Computacionales}
	%\title{\Huge \colorbox{magenta}{Taller de Herramientas computacionales}} %De esta forma con colorbox pone el texto dentro de una "caja" de color.
	\author{Josué Artemio Hernández Rodríguez}%autor del escrito
	\date{23/Enero/19}%fecha del escrito

\begin{document}%inicia el documento
\maketitle
%\vfill %Para rellenar el espacio y colocar hasta abajo de la pagina el siguiente texto, imagen.
\begin{center}%inicia centrado
\includegraphics[scale=.6]{3.jpg}%del lado izquierdo se muestra el tamaño de la imagen, del derecho se escribe el nombre de la imagen a incluir en el texto
\end{center}%termina el centrado para la imagen
\newpage%crea una nueva página

\title{\huge Preguntas: 23 de enero de 2019\\}%titulo2 \\ sirven para saltar una linea.


\begin{enumerate}%Inicio de númeración para enlistar las cosas vistas en clase.
	\item ¿Qué sucede al colocar un comentario entre ''' ''' después de definir una función? \\
	 este aparece en el shell como una nota de ayuda o información emergente. La puedes utilizar para describir que hace la función, y que tipo de valores pueden tomar las variables por ejemplo.
	
	\item ¿Cómo se les llama a las funciones que no tienen retorno?\\
	Se les denomina rutinas o procedimientos; y en vez de retornar algo, este directamente imprime.
	\item ¿Cuales son los dos valores de validez que existen en python, y cual es la diferencia?\\
	Son las variables locales y las globales. LA diferencia es que los globales estan fuera, y pueden ser ocupadas por todas la funciones, mientras que las locales estan dentro de una funcion, y solo es parte de ella, y no de las demás.
	
	\item ¿Qué sucede en la compilacion de los siguientes codigos, que observaciones puedes dar?\\
	\begin{lstlisting}
1.-L = []
   if L:
       print "hola"
	
	
2.- L = [1,2,3]
    if L:
        print "hola mundo" 
	\end{lstlisting}
	La primera al evaluar la lista "L" con if verifica si es True o False, y como esta vacía el if la tomara como False y no va hacer nada, mientras que la segunda como tiene elementos, la tomara como True y va imprimir "hola mundo"

	\item Menciona que tipo de funcion es y como funciona \begin{lstlisting}
def suma(n):
   if n>1:
       return n + suma(n-1)
   else:
       return (1)
	\end{lstlisting}
	
	Es una función recursiva, toma como caso base cuando n es igual a 1, entonces retorna 1. Entonces a partir de ahi, si le das una n mayor a 1 calculara los anteriores y los va sumando, hasta tener la suma de los n primero naturales.
	
	 
	
	
	
	
	
\end{enumerate}%finaliza el enlistado principal


\end{document}%termina el documento