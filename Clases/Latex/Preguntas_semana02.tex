\documentclass[letterpaper, 12pt, oneside]{article}%especificaciones del documento
\usepackage{amsmath}%paquete para escribir expresiones matemámaticas
\usepackage{graphicx}%paquete para poder incluír imagenes en el documento
\usepackage{xcolor} %paquete de LaTex para poder poner otro texto
\graphicspath{{Imagenes/}}%directorio de la imagen, este lo cambian por el directorio en el que ustedes guardaron su imagen 1.png
\usepackage[utf8]{inputenc} %para poder poner acentos
\usepackage{listings}

	\title{\Huge Taller de Herramientas Computacionales}
	%\title{\Huge \colorbox{magenta}{Taller de Herramientas computacionales}} %De esta forma con colorbox pone el texto dentro de una "caja" de color.
	\author{Josué Artemio Hernández Rodríguez}%autor del escrito
	\date{13/01/19}%fecha del escrito

\begin{document}%inicia el documento
\maketitle
%\vfill %Para rellenar el espacio y colocar hasta abajo de la pagina el siguiente texto, imagen.
\begin{center}%inicia centrado
\includegraphics[scale=0.2]{2.png}%del lado izquierdo se muestra el tamaño de la imagen, del derecho se escribe el nombre de la imagen a incluir en el texto
\end{center}%termina el centrado para la imagen
\newpage%crea una nueva página

\title{\huge Preguntas: Semana II\\}%titulo2 \\ sirven para saltar una linea.

La semana 2 se resume en las siguientes preguntas:%letras de color azul
\begin{enumerate}%Inicio de númeración para enlistar las cosas vistas en clase.
	\item ¿Qué es un algoritmo? \\
	 Es un conjunto finito de instrucciones para resolver
	 un problema especifico.
	
	\item ¿Cúal consideras que es la diferencia fundamental entre igualdad y asignación?\\
	Una igualdad puede despejarse y encontrar el valor, y una asignación no.
	\item ¿A que nos referimos con "flavors de linux"?\\
	A las diferentes versiones de una distribución de linux, y que en la mayoria de los casos son versiones mas ligeras
	
	\item ¿Que son y para que se utilizan if y else?\\
	Son condicionales de python, si if es verdadera ejecuta lo que esta dentro del bloque, y else realiza algo mientras if sea falso.
	\item ¿Cuales son las partes globales de un codigo para reolver un problema?
		\begin{enumerate}
			\item Definir valores iniciales
			\item Partes que procesa la información
			\item Y la parte del resultado
		\end{enumerate}
	
	\item ¿Cúal es la diferencia entre if y while?\\
	While se ejecuta hasta que se cumpla una condicion, en un número finito de veces, e if si su condicion es verdadera realiza lo que esta dentro del bloque una sola vez, si no es verdadero, se va a else.
	
	\item ¿Como abres una seccion nueva sin un número que la identifique en \LaTeX\\
	De la misma forma solo que se le coloca un asterisco al final, es decir: \begin{lstlisting}
\section*{title}
	\end{lstlisting}
	
	\item ¿Cómo centras una fórmula matemática en \LaTeX?\\
	Se usan los corchetes de esta manera:  \begin{lstlisting}
\[ Formula   \]
	\end{lstlisting}
	
	\item ¿En el contexto de python, a que nos refermimos con un "metodo"?\\
	Son las acciones que realiza un objeto, modifican un
	objeto o hace que interaccione con otro
	
	\item ¿Cómo pones en pausa un proceso?\\
	Si se abre un programa desde la terminal, queda en espera, entonces es aqui donde se hace la siguiente combinación : ctrl + z
	
	\item ¿Para que srive esta combinacion: "programa" \&?
	Arranca la aplicación ”programa” pero deja li-
	bre la terminal y asi se evita tener varias ventanas de la termnal
	abiertas
	\item ¿Qué accion realiza el comando kill y kill -9?\\
Kill manda llamar a un proceso, pero no le hace nada, pero si le agregaos -9, este se cierra.
	\item ¿Qué realiza el siguiente comando chmod +x ”archivo?\\
		Le asigna atributos de ejecucion a un archivo
		
	\item ¿Como buscas la ruta de un programa en especifico?\\
	Con where is ”programa"
	
	\item ¿Cómo buscarias todos los archivos con la extension .pdf?\\
	Con el comando siguiente dentro de la terminal: find . -name ”*.pdf”
	
	\item ¿Cual es la diferencia entre ”x,y,z” y x,y,z=?\\
Que uno es una cadena con el texto x,y,z pero el otro es una asignación de valores

	\item ¿Cómo colocas una matris en un documento de \LaTeX?\\
	\\
	Con el codigo siguiente:
\begin{verbatim} 
\begin{bmatrix}
x {valor1} & x {valor2}\\
x {valor3} & x {valor4}\\
\end{bmatrix}
\end{verbatim}


	\item Menciona tres formas de colocar puntos suspencivos en LaTeX\\
	Se usa dots para los puntos suspensivos normales, para los demas hay una variacion en la letra inicial, es decir:
\begin{verbatim} 
	\dots : Coloca puntos de forma horizontal
	\ddots : Coloca puntos de manera diagonal
	\vdots : Coloca puntos de manera vertical
\end{verbatim}	
	\item ¿Por qué es importante usar una variable auxiliar para intercambiar el valor de una variable por la de otra?\\
	Cuando se necesite cambiar el valor de una variable a otra, es
de suma importancia usar un auxiliar para que el valor que tenia
antes no se pierda.

	\item En python, ¿que significa lo siguiente: a\%b = c, r != 0?\\
La primera expresion Nos dice que r es diferente de cero y en la siguiente, el operador modulo nos da como resultado el residuo de la división.
	\item ¿Para qué nos sirve importar comandos de OS a python?\\
Se pueden importar comandos procedentes de sistema operativo a python, y
de esta forma tener de forma clara en que directorio nos encon-
tramos, o movernos de directorio si es necesario. Esto nos sirve
para movernos donde esta un modulo en especifico y asi importar
una función.
	\item ¿Cómo realizas un documento de tipo libro en LaTex?
	Con documentclass{book}
	\item ¿Cómo le colocas bibliografia a tu documento de LaTex?
	Con el paquete de usepackage{biblatex} y para agregar la referencia bibliografica usamos: \begin{verbatim}
\begin{thebibliography}{#}
\bibitem{Libro}
	\end{verbatim}
	\item ¿Como dibijas una tabla?
\begin{verbatim}
Con el comando siguiente: \begin{array}{|v|v|v|}
                          Texto" & "texto & "texto
                          \end{array}
\end{verbatim}
	\item ¿Para que sirve "tableofcontents?
	Agrega un índice de contenido en nuestro documento
	
	\item ¿Para qué nos sirven las lista en python?\\
	Las listas en python son una herramienta muy interesante y potente que nos sirve para recolentar datos de manera robusta, estos datos adquieren la cualidad de objetos dentro de la lista
	\item ¿Cúal es la diferencia entre posición e índice?\\
	La diferencia de posición e índice es que la posición es un número n del 1 a un numero fínito, y el índice es (n-1)
	\item ¿Cómo haces para abrir dos programas y que estos no anulen nuestra terminal?
	Con una separacion usando \&, es decir: program1 \& program2
	\item ¿Para que nos sirve el comando siguiente en python: type()?\\
	Nos dice que tipo de dato es una variable, si es entero, flotante, cadena, etc.
	\item ¿Para que nos sirve variable.append?\\
	Agrega un elemento después del último dentro de una lista
	\item ¿Como sabes por cuantos elementos esta conformada una lista?\\
	Con len(a), donde a es la variable con la cual esta guardada la lista.
	\item ¿Qué significa lo siguiente: L.insert(3,”otra cadena”)?
	Nos dice que en el índice 3 agrega el elemento ”otra cadena”
	\item ¿Cúal es la diferencia entre L[3] y Len(L[3])?
	Que el primero nos indica cual es el elemento con indice 3 en la lista y el otro nos dice cual es la longitud(cuantos elementos tiene) del elemento con índice 3.
	\item ¿Para qué nos sirve V.pop()?\\
	Saca un elemento de la lista V. Si no le especifico un índice dentro de () sacara el último elemento de la lista.
	\newpage
	\item Menciona que hacen los siguientes codigos de python
	\begin{verbatim}
1.- L[3] + L[len(L)-1] 
	\end{verbatim}
	Junta los elementos con indice 3 y el ultimo de la lista en uno solo, esdecir si L[3]= Buenos y L[len(L)-1] = días, dentro de la lista:
	 [1,2,3,"¿Cómo,45,62,"estás?"], nos da como resultado, "Buenosdías"
	\begin{verbatim}
2.- for C in gradosC:
        print 'Elemento de la lista: ', C
	\end{verbatim}
	Por cada indice en la lista gradosC, imprime "Elemento de la lista: + el elemento perteneciente a cada indice" 
	\begin{verbatim}
3.- print 'La lista C tiene: ',len(C), 'elementos'
	\end{verbatim}
		Nos imprime de cuantos elementos esta conformada la lista C.
	
	
\end{enumerate}%finaliza el enlistado principal


\end{document}%termina el documento