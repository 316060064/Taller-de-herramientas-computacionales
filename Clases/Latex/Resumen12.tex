\documentclass[letterpaper, 12pt,oneside]{article}
\usepackage{amsmath}
\usepackage{graphicx}
\usepackage{xcolor}
\graphicspath{{Imagenes/}}
\usepackage[utf8]{inputenc}
\usepackage{listings}

\title{\Huge Taller de Herramientas Computacionales}
\author{Josué Artemio Hernández Rodríguez}
\date{22/Enero/2019}

\begin{document}
	\maketitle
	\begin{center}
		\includegraphics[scale=0.2]{2.png}
	\end{center}

	\newpage
	
	\title{\huge \textit{El uso de listas en Python III}}\\
	
	El resumen del día martes 22 de Enero de 2019 abarca lo siguiente:
	\begin{enumerate}
		\item Conceptos
			\begin{itemize}
				\item Las sublistas sirven para analizar informacion en un rango.
				\item Se puede decir que las listas en python son listas circulares, porque si expecificas un índice negativo toma valores de derecha a izquierda.
				\item Referencia, es cuando le asignamos el valor de una variable a otra.
				\item Los atributos y metodos estan alojados en la memoria, y estan sumamente relacionados con los objetos.
				\item Las variables son identificadores de un valor especifico
				\item Se puede crear una copia de una lista, puede resultar que sus elemntos sean los mismos, pero como objeto no es el mismo.
				\item La diferencia de \textit{for i in range():} y \textit{for i in lista:} es que la primera toma en cuenta los índices y la segunda forma toma la posición.
			\end{itemize}
		
		
		
		\item Código de python
			\begin{itemize}
				\item {\textit{A[2:]}} : Regresa valores a partir del indice 2
				\item {\textit{A[1:3]}} : Pone como rango de elementos a mostrar al índice 1 y 3, pero no se muestra el 3, digamos que es un intervalo cerrado/abierto.
				\item {\textit{A[:3]}} : Muestra todos los valores que pertenecen a cada indice en A, pero como limite al 3, sin mostrarlo
				\item {\textit{[1:-1]}} : Regresa todo, excepto el primero y el ultimo.
			
				\item {\textit{tabla[4:7][0:2]}} : De la lista que crea forma otra con el rango que se le esta indicando.
					
				\item {\textit{lista.index(\#)}} : Busca dónde está el valor que le especifiquemos (\#) y te regresa en que índice se encuentra
			
				\item \textit{lista.sort()} : Ordena los valores de la lista de menor a mayor
				\item \textit{B = A[:]} : Crea una sublista de B, que a la vez es una copia fiel de cada argumento dentro de A.
				\item \textit{B == A} : Si tienen los mismos elementos A y B, regresa un \textit{True}.
				\item \textit{B is A} : Regresa un \textit{True} si es el mismo objeto
				
			\end{itemize}
		
		
		\item Comandos de Latex
		\begin{itemize}
			\item \begin{lstlisting}
\documentclass{beamer}
			\end{lstlisting} Este tipo de documento se utiliza para crear presentaciones.
			\item \begin{lstlisting}
\begin{frame}
	\end{frame}
			\end{lstlisting} : Indica que aqui inica una nueva diapositiva
			\item \begin{lstlisting}
\frametitle{title}
			\end{lstlisting} Le asigna un título a la diapositiva
			\item \begin{lstlisting}
\transboxin
\transblindshorizontal
			\end{lstlisting} Le agrega efectos de transición a cada diapositiva.
			\item \begin{lstlisting}
\begin{frame}[fragile]
			\end{lstlisting} Se utiliza para especificar que en esa zona se colocara un \textit{verbatin} para escribir código de programación de python u otros.
			
			
			
		\end{itemize}
		
	\end{enumerate}
	
	
	
	
	
	
	
	
	
	
	
\end{document}
